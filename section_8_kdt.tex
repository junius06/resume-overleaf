% YAAC Another Awesome CV LaTeX Template
%
% This template has been downloaded from:
% https://github.com/darwiin/yaac-another-awesome-cv
%
% Author:
% Christophe Roger
%
% Template license:
% CC BY-SA 4.0 (https://creativecommons.org/licenses/by-sa/4.0/)
%Section: Work Experience at the top
\sectionTitle{K-Digital-Training}{\faSuitcase}
%\renewcommand{\labelitemi}{$\bullet$}

%%%%% KOREAN %%%%%
\begin{K-Digital-Training}
  \experience
    {재직자 과정} {재직자 Skill Up 클라우드 전문가}{멀티캠퍼스}{NCS(20010215)}
    {2024.10-2024.12} {
                      \begin{itemize}
                        \item 보안·이중화·확장성을 고려한 AWS 기반 클라우드 인프라 아키텍처 설계
                        \item 컨테이너·MSA·CI/CD 기반 클라우드 네이티브 인프라 전환 및 구축
                        \item Ansible/IaC 기반 인프라 표준화 및 자동화 운영 환경 구현
                      \end{itemize}
                    }
                    {AWS, Tomcat, Docker, EKS, ArgoCD, Github Actions, Terraform, Grafana, Prometheus, MySQL}
  \emptySeparator
  \experience
    {구직자 과정} {쿠버네티스 전문가 양성 과정}{주식회구름}{NCS(20010213)}
    {2022.12-2023.04} {
                      \begin{itemize}
                        \item Linux 시스템 운영 및 네트워크 기초, AWS 인프라 설계·운영
                        \item Docker·Kubernetes 기반 컨테이너 애플리케이션 구축 및 운영
                        \item CI/CD 파이프라인 구축, 무중단 배포 및 오토스케일링 적용
                        \item Ansible 기반 코드형 인프라(IaC) 및 AWS 리소스 자동화
                        \item 모니터링·알람 시스템 구축 및 서비스 안정성 관리
                      \end{itemize}
                    }
                    {AWS, CDN, Docker, EKS, ArgoCD, Github Actions, Ansible, Terraform, Grafana, Prometheus, MySQL, Redis}
  \emptySeparator
  \experience
    {재직자 과정} {데브옵스 자동화를 위한 Ansible}{솔데스크}{NCS(20010301)}
    {2021.10-2021.11}    {
                      \begin{itemize}
                        \item Ansible 플레이북·Role 기반 인프라 자동화 구현
                        \item Jinja2 템플릿을 활용한 설정 관리
                        \item 작업 제어, 오류 처리, Vault를 통한 보안 자동화
                        \item Vagrant 기반 DevOps 환경 구성 및 배포 실습
                      \end{itemize}
                    }
                    {Ansible}
  \emptySeparator
    \experience
    {구직자 과정} {클라우드 네트워크 시스템 보안 운영관리}{패스트캠퍼스강북학원}{NCS(20020103)}
    {2019.06-2020.03}    {
                      \begin{itemize}
                        \item 정보보안 (보안위협 관리통제, 물리적 보안 구축)
                        \item 정보시스템운영 (SQL활용, 데이터베이스 구현)
                        \item 정보시스템구축 (애플리케이션 테스트 수행, 화면 구현, 응용SW 기초기술 활용, 프로그래밍 언어활용, 애플리케이션 배포)
                        \item 네트워크운영 (네트워크보안관리, 네트워크유지보수, 네트워크운용관리, 네트워크공사발주)
                        \item 네트워크구축 (서버 구축, 무선랜 구축, L2·L3 스위치 구축, 근거리통신망(LAN) 설계)
                      \end{itemize}
                    }
                    {CentOS, Network}
  \emptySeparator
\end{K-Digital-Training}
