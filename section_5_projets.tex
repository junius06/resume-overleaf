% Awesome Source CV LaTeX Template
%
% This template has been downloaded from:
% https://github.com/darwiin/awesome-neue-latex-cv
%
% Author:
% Christophe Roger
%
% Template license:
% CC BY-SA 4.0 (https://creativecommons.org/licenses/by-sa/4.0/)

%Section: Project
\sectionTitle{Projets}{\faLaptop}
% 성과와 실전 문제 해결 사례 등 기술

\begin{projects}

    %%%%%%%%%%%% KOREAN %%%%%%%%%%%%
    \project
	{삼성전자 MX FMM 프로젝트 EKS 운영}{SKTelecom | 2025.04 - Present}
	{\website{https://smartthingsfind.samsung.com/}{https://smartthingsfind.samsung.com/}}
	{
    • AWS VM 기반 3-tier 환경을 EKS로 마이그레이션 및 운영 안정화\\
    • CI/CD 및 모니터링 표준화로 배포 안정성 향상\\
    • ArgoCD 기반 GitOps 적용으로 애플리케이션 배포 자동화 및 환경 표준화\\
    • OpenTelemetry 기반의 Observability 환경 구축 및 Logs, Metrics, Traces 데이터 수집 및 분석으로 시스템 가시성 향상과 장애 원인 분석 효율 개선
    • EKS Access Entry 기반 사용자·권한 분리 적용하여 클러스터 접근 제어 체계화
    }
	{GitOps, AWS, Terraform, Helm, CloudWatch, Grafana, Prometheus, Loki, Promtail, OpenTelemetry, Tempo, KeyCloak, AWS Access Entry}
    \vspace{0.7em}
    
    \project
	{(THEPOL) 인프라 환경 인수 및 AWS 기반 재구축·고도화}{CPLABS | 2024.09 - 2024.12}
	{\website{https://thepol.com/}{https://thepol.com/}}
	{
    • 외주 업체 관리 인프라 환경 인수 및 내부 전환 수행\\
    • AWS 기반 Container 인프라 재구축 및 운영 안정화\\
    • CI/CD 파이프라인 리팩토링을 통한 중복 코드 제거 및 image 경량화로 빌드 시간 단축\\
    • Github CI → Bitbucket Pipielines 이전을 통해 관리 포인트 통합\\
    • 신규 모니터링 환경 구축하여 시스템 가시성 및 장애 분석 고도화
    }
	{AWS, Docker, Python, Shell Script, CloudWatch, Grafana, Prometheus, Promtail, BlackBox, Gitlab CI, Bitbucket Pipelines, PostgreSQL}
    \vspace{0.7em}

    \project
	{AI Telegram OJT Bot}{CPLABS | 2024.08 - 2024.09}
	{}
	{
    • 신입 온보딩 자동화를 위한 AI 기반 Telegram Bot 개발\\
    • 반복 문의 감소 및 운영 리소스 절감
    }
	{Python, Telegram, ChatGPT}
    \vspace{0.7em}
    
    \project
	{(WEB2X) 애플리케이션 기반 인프라 설계 및 DevOps 환경 구축}{CPLABS | 2024.05 - 2024.12}
	{\website{https://cplabs.io/web2x}{https://cplabs.io/web2x}}
	{
    • 신규 플랫폼 인프라 구축과 함께 사내 플랫폼 전반의 인프라 구성 기준을 표준화하여, 운영 효율성과 확장성을 고려한 통합 인프라 환경을 수립
    • 개발 애플리케이션 요구사항 분석을 기반으로 인프라 아키텍처 설계\\
    • AWS 클라우드 환경에서 서버, 컨테이너 인프라 신규 구축 및 운영\\
    • CI/CD 파이프라인 구축으로 개발-배포 프로세스 자동화\\
    • 서비스 상태 및 장애 분석을 위한 모니터링 환경 신규 구성\\
    • 애플리케이션과 MySQL 간 네트워크,  보안 설정 및 통신 구성\\
    • Jump Server 구축을 통한 보안 접근 체계 수립 및 개발자 접근 권한 제공\\
    }
	{AWS, Docker, Python, Shell Script, Ansible, CloudWatch, Grafana, Prometheus, Promtail, BlackBox, Bitbucket Pipelines, MySQL}
    \vspace{0.7em}
    % 여기가 Redis 사용 그거였나???
    
	\project
	{(WEMIX) 클라우드 인프라 AKS 마이그레이션}{CPLABS | 2023.10 - 2024.05}
	{\website{https://www.wemix.com}{https://www.wemix.com}}
	{
    • VM 기반 서비스를 Kubernetes 환경으로 이전\\
    • CI/CD 자동화 및 민감 정보 관리 체계 개선\\
    • 배포 실패율 감소 및 운영 자동화 수준 향상
    }
	{Azure, Docker, AKS, Python, Shell Script, ArgoCD, Github Actions, Bitbucket Pipelines, CloudWatch, Grafana, Prometheus, Promtail, BlackBox, Azure Key Vault, Ansible, node.js}
    \vspace{0.7em}

    \project
	{(WEMIX) 클라우드 인프라 운영 및 모니터링}{CPLABS | 2023.4 - 2024.05}
	{}
	{
    • Azure 클라우드 VM 환경 운영 및 모니터링
    }
	{Azure, Shell Script, pm2, node.js, Zabbix}
    \vspace{0.7em}
				
	\project
	{더존비즈온 그룹웨어 연계솔루션}{고은정보기술 | 2020.06 - 2022.06}
	{}
	{
    • 고객사 및 IDC 방문하여 하드웨어 \& 솔루션(Anti-Spam / Veritas NetBackup) 설치\\
    • Anti-Spam / Vertas NetBackup 교육 및 유지보수\\
    • 그룹웨어 도메인을 위한 SSL 발급 및 적용
    }
    {CentOS, Network, DNS, SSL}

    %%%%%%%%%%%% ENGLISH %%%%%%%%%%%%

\end{projects}